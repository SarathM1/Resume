%%%%%%%%%%%%%%%%%
% This is an sample CV template created using altacv.cls
% (v1.6, 21 May 2021) written by LianTze Lim (liantze@gmail.com). Now compiles with pdfLaTeX, XeLaTeX and LuaLaTeX.
%
%% It may be distributed and/or modified under the
%% conditions of the LaTeX Project Public License, either version 1.3
%% of this license or (at your option) any later version.
%% The latest version of this license is in
%%    http://www.latex-project.org/lppl.txt
%% and version 1.3 or later is part of all distributions of LaTeX
%% version 2003/12/01 or later.
%%%%%%%%%%%%%%%%

%% Use the "normalphoto" option if you want a normal photo instead of cropped to a circle
% \documentclass[10pt,a4paper,normalphoto]{altacv}

\documentclass[10pt,a4paper,ragged2e,withhyper]{altacv}
%% AltaCV uses the fontawesome5 and packages.
%% See http://texdoc.net/pkg/fontawesome5 for full list of symbols.

% Change the page layout if you need to
\geometry{left=1.25cm,right=1.25cm,top=1.5cm,bottom=1.5cm,columnsep=1.2cm}

% The paracol package lets you typeset columns of text in parallel
\usepackage{paracol}

% Change the font if you want to, depending on whether
% you're using pdflatex or xelatex/lualatex
\ifxetexorluatex
% If using xelatex or lualatex:
\setmainfont{Roboto Slab}
\setsansfont{Lato}
\renewcommand{\familydefault}{\sfdefault}
\else
% If using pdflatex:
\usepackage[rm]{roboto}
\usepackage[defaultsans]{lato}
% \usepackage{sourcesanspro}
\renewcommand{\familydefault}{\sfdefault}
\fi

% Change the colours if you want to
\definecolor{SlateGrey}{HTML}{2E2E2E}
\definecolor{LightGrey}{HTML}{666666}
\definecolor{DarkPastelRed}{HTML}{450808}
\definecolor{PastelRed}{HTML}{8F0D0D}
\definecolor{GoldenEarth}{HTML}{E7D192}
\colorlet{name}{black}
\colorlet{tagline}{PastelRed}
\colorlet{heading}{DarkPastelRed}
\colorlet{headingrule}{GoldenEarth}
\colorlet{subheading}{PastelRed}
\colorlet{accent}{PastelRed}
\colorlet{emphasis}{SlateGrey}
\colorlet{body}{LightGrey}

% Change some fonts, if necessary
\renewcommand{\namefont}{\Huge\rmfamily\bfseries}
\renewcommand{\personalinfofont}{\footnotesize}
\renewcommand{\cvsectionfont}{\LARGE\rmfamily\bfseries}
\renewcommand{\cvsubsectionfont}{\large\bfseries}


% Change the bullets for itemize and rating marker
% for \cvskill if you want to
\renewcommand{\itemmarker}{{\small\textbullet}}
    \renewcommand{\ratingmarker}{\faCircle}

    %% Use (and optionally edit if necessary) this .tex if you
    %% want to use an author-year reference style like APA(6)
    %% for your publication list
    % When using APA6 if you need more author names to be listed
% because you're e.g. the 12th author, add apamaxprtauth=12
\usepackage[backend=biber,style=apa6,sorting=ydnt]{biblatex}
\defbibheading{pubtype}{\cvsubsection{#1}}
\renewcommand{\bibsetup}{\vspace*{-\baselineskip}}
\AtEveryBibitem{\makebox[\bibhang][l]{\itemmarker}}
\setlength{\bibitemsep}{0.25\baselineskip}
\setlength{\bibhang}{1.25em}


    %% Use (and optionally edit if necessary) this .tex if you
    %% want an originally numerical reference style like IEEE
    %% for your publication list
    % \usepackage[backend=biber,style=ieee,sorting=ydnt]{biblatex}
%% For removing numbering entirely when using a numeric style
\setlength{\bibhang}{1.25em}
\DeclareFieldFormat{labelnumberwidth}{\makebox[\bibhang][l]{\itemmarker}}
\setlength{\biblabelsep}{0pt}
\defbibheading{pubtype}{\cvsubsection{#1}}
\renewcommand{\bibsetup}{\vspace*{-\baselineskip}}


    %% sample.bib contains your publications
    \addbibresource{sample.bib}


\begin{document}
    \name{Sarath M}
    \tagline{Machine Learning Engineer}
    %% You can add multiple photos on the left or right
    %\photoR{2.8cm}{Globe_High}
    % \photoL{2.5cm}{Yacht_High,Suitcase_High}

    \personalinfo{%
        % Not all of these are required!
        \email{sarathm00@gmail.com}
        \phone{+91 9496354518}
        \mailaddress{Saranya House, North Paravoor P.O, 683513}
        \location{Kerala, India}
        %\homepage{www.homepage.com}
        %\twitter{@twitterhandle}
        %\linkedin{your_id}
        \github{SarathM1}
        %\orcid{0000-0000-0000-0000}
        %% You can add your own arbitrary detail with
        %% \printinfo{symbol}{detail}[optional hyperlink prefix]
        % \printinfo{\faPaw}{Hey ho!}[https://example.com/]
        %% Or you can declare your own field with
        %% \NewInfoFiled{fieldname}{symbol}[optional hyperlink prefix] and use it:
        % \NewInfoField{gitlab}{\faGitlab}[https://gitlab.com/]
        % \gitlab{your_id}
        %%
        %% For services and platforms like Mastodon where there isn't a
        %% straightforward relation between the user ID/nickname and the hyperlink,
        %% you can use \printinfo directly e.g.
        % \printinfo{\faMastodon}{@username@instace}[https://instance.url/@username]
        %% But if you absolutely want to create new dedicated info fields for
        %% such platforms, then use \NewInfoField* with a star:
        % \NewInfoField*{mastodon}{\faMastodon}
        %% then you can use \mastodon, with TWO arguments where the 2nd argument is
        %% the full hyperlink.
        % \mastodon{@username@instance}{https://instance.url/@username}
    }

    \makecvheader
    %% Depending on your tastes, you may want to make fonts of itemize environments slightly smaller
    % \AtBeginEnvironment{itemize}{\small}

    %% Set the left/right column width ratio to 6:4.
    \columnratio{0.6}

    % Start a 2-column paracol. Both the left and right columns will automatically
    % break across pages if things get too long.
    \begin{paracol}{2}
        \cvsection{Experience}
            \cvevent{Specialist}{Tata Elxsi}{July 2016 -- Ongoing}{Technopark, Trivandrum}
            %\begin{itemize}
            %\item Job description 1
            %\item Job description 2
            %\end{itemize}

            \divider

            \cvevent{Embedded System Engineer}{Unisync Technologies}{Jan 2015 -- July 2016}{Vyttila, Ernakulam}
            %\begin{itemize}
            %\item Job description 1
            %\item Job description 2
            %\end{itemize}
        \cvsection{Most Proud of}
            \cvachievement{\faTrophy}{My Professional Acheievment}{Awarded the highest rating ”Outstanding” in three consecutive appraisal cycles in Tata Elxsi}

            \divider

            \cvachievement{\faTrophy}{My Academic Achievement}{Final year academic project "Hexapod" was selected for the finals in State level competition}

            \divider

            \cvachievement{\faHeartbeat}{Martial Arts}{Black Belt holder in Shito-Ryu style of Karate}


            \medskip
        \cvsection{Education}
            \cvevent{Course in Embedded Systems}{Vector India Institute, Bangalore, Karnataka}{2014}{}
            \divider

            \cvevent{B.Tech (ECE)}{Govt. College of Engineering Cherthala,\\Cochin University Of Science and Technology, Kerala}{2010 - 2014}{}

            \medskip
        \cvsection{Interests}

            % Adapted from @Jake's answer from http://tex.stackexchange.com/a/82729/226
            % \wheelchart{outer radius}{inner radius}{
            % comma-separated list of value/text width/color/detail}
            % This is taken from AltaCV
            % see https://github.com/liantze/AltaCV for details
            \wheelchart{1.5cm}{0.5cm}{% outer and inner diameter
                6/8em/accent!20/Podcasts \& Audio books,
                2/8em/accent!40/Budget Tracking,    % fraction of 24 / line length / color / label
                8/8em/accent!80/Automation using Python,          % here, the color is shades of the accent color
                3/8em/accent!60/Stock Market,
                5/8em/accent/UI Development,
                9/8em/accent!30/Machine Learning
                }

                \switchcolumn

        \cvsection{My Life Philosophy}
            \begin{quote}
                ``Quality is not an act; it is a habit.''
            \end{quote}

        \cvsection{Strengths}
            \cvtag{Team Player}
            \cvtag{Passionate Programmer}
            \cvtag{Fast Learner}
            \cvtag{Hard-working}
            \cvtag{Eye for detail}\\

            %\divider\smallskip

            %\cvtag{C++}
            %\cvtag{Embedded Systems}\\
            %\cvtag{Statistical Analysis}
        \cvsection{Skills}
            \begin{itemize}
                \item Machine Learning Frameworks
                    \smallskip

                    \cvskill{Tensorflow}{4}
                    \cvskill{Scikit-learn}{5}
                    \cvskill{NLTK}{3}
                    \divider

                \item Visualisation \& Data Processing
                    \smallskip

                    \cvskill{Pandas}{5}
                    \cvskill{Numpy}{4}
                    \cvskill{Flask}{5}
                    \cvskill{Bokeh}{5}
                    \cvskill{Plotly}{4}
                    \cvskill{Grafana}{5}
                    \cvskill{Twitter Bootstrap}{5}
                    \cvskill{PyQt4, Qt Designer}{5}
                    \divider

                \item Distributed Systems
                    \smallskip

                    \cvskill{ROS}{4}
                    \cvskill{Paho MQTT}{5}
                    \cvskill{Redis}{4}
                    \cvskill{Apache Kafka}{3}
                    \divider

                \item Test Automation \& CI
                    \smallskip

                    \cvskill{Jenkins}{4}
                    \cvskill{Robot Framework}{5}
                    \divider

                \item Devops
                    \smallskip

                    \cvskill{Docker}{4}
                    \cvskill{Ansible}{4}
            \end{itemize}


            %Other tools \& frameworks
            %\cvskill{Kafka}{5}
            %\cvskill{ROS}{5}
            %\cvskill{MQTT}{5}
            %\cvskill{Redis}{5}
            %\cvskill{Influxdb}{4}
            %\divider

            %Development Tools
            %Git, SVN, Zsh, Vim, Emacs,\\VS Code, Jira

            %% Yeah I didn't spend too much time making all the
            %% spacing consistent... sorry. Use \smallskip, \medskip,
            %% \bigskip, \vspace etc to make adjustments.
            \medskip


            % \divider
         \cvsection{Languages}

            \cvskill{English}{5}
            % \divider

            \cvskill{Malayalam}{5}
            % \divider

            \cvskill{Hindi}{3}
            %\cvsection{Referees}

            %% \cvref{name}{email}{mailing address}
            %\cvref{Prof.\ Alpha Beta}{Institute}{a.beta@university.edu}
            %{Address Line 1\\Address line 2}

            %\divider

            %\cvref{Prof.\ Gamma Delta}{Institute}{g.delta@university.edu}
            %{Address Line 1\\Address line 2}
    \end{paracol}

    % use ONLY \newpage if you want to force a page break for
    % ONLY the current column
    \newpage

    \cvsection{Projects}
        \cvevent{ADAS features for Autonomous Vehicle project}{}{}{}
            The objective of the project is to develop an L2 Autonomous Car
            \linebreak
            \begin{itemize}
                \item Development of Object detection system using Convolutional Neural Networks
                    \begin{itemize}
                        \item Faster-RCNN, Yolo V2, Single Shot Detectors
                    \end{itemize}
                \item Development of Drivable Area using Image Segmentation
                    \begin{itemize}
                        \item SegNet, Mask R-CNN
                    \end{itemize}
                \item Design and devlopment of Distributed System using Robotic Operating System (ROS)
                \item Testing and deployment of the ML model in NVidia Jetson TX1 platform
                \item Object detection in 3D Point cloud data using PointNet
            \end{itemize}
            \divider

        \cvevent{Lidargen}{}{}{}
            The objective of the project was to create a tool that would simulate the actual output from a LIDAR sensor. The tool takes three dimensional CAD objects as input and generates corresponding pointcloud.
            \linebreak
            \begin{itemize}
                \item The User is given the freedom to place and orient the multiple meshes
                \item Used in the optimised placement of multiple Lidars to reduce blind spot
                \item To mathematical model the behavior of Lidar and simulate the same using Ray Casting.
                \item A UI capable of 3D visualization was created using Qt and ROS RViz
            \end{itemize}
            \divider


        \cvevent{Intelligent Battery Management System}{}{}{}
            The objective of the project is to estimate State of Health (SOH) and State of Charge (SOC) of a Lithium Ion battery using Machine Learning
            \begin{itemize}
                \item Implemented Neural Network Regression model as a benchmark
                \item Implemented LSTM models to improve upon the bencharm results
                \item Supported in development of 'Digital Twin' of a cell with ML models mimicing the electrochemical characteristics
                \item Developed a POC on Anomaly detection algorithm to demonstrate online-learning capabilities of the framework
                \item Create a Data dashboard for monitoring sensor data in real-time using Flask and Bokeh
            \end{itemize}
            \divider

        \cvevent{UAV Based Driver View Enhancement}{}{}{}
            \begin{itemize}
                \item  To develop a proof of concept, using a Drone to enhance the view of the Driver and cover blind spots for a larger area around the vehicle

                \item  Designing a State Machine to control the Drone, Create a web application as a User interface using Python as a backend
            \end{itemize}
            \divider

        \cvevent{Automated Testing and CI Framework}{}{}{}
            A Test automation framework was developed with the following key features \linebreak
            \begin{itemize}
                \item Enables rapid and Automated testing
                \item The framework supports both SIL and HIL testing
                \item Continuous Integration using Jenkins, Ansible and Robot framework
                \item Centralized Logging using Fluentd, Elasticsearch and Kibana
                \item Containerization using Docker
            \end{itemize}
\end{document}
