\documentclass[%singlesided,
               doublesided,
               paper=a4,
               fontsize=10pt
              ]{my-resume}


%%%%%%%%%%%%%%%%%%%%%%%%%%%%%%%%%%%%%%%%%%%%%%%%%%%%%%%%%%%%%%%%%%%%%%%%%%%%%%%%
% set geometry
%%%%%%%%%%%%%%%%%%%%%%%%%%%%%%%%%%%%%%%%%%%%%%%%%%%%%%%%%%%%%%%%%%%%%%%%%%%%%%%%

\setlength\highlightwidth{8cm}
\setlength\headerheight{1.5cm}            % note that margintop gets added to this value, i.e. the header bar is 5cm
\setlength\marginleft{1cm}
\setlength\marginright{\marginleft}      % needs to be 1.5 times to be actually equal. why?
\setlength\margintop{1cm}
\setlength\marginbottom{1cm}


%%%%%%%%%%%%%%%%%%%%%%%%%%%%%%%%%%%%%%%%%%%%%%%%%%%%%%%%%%%%%%%%%%%%%%%%%%%%%%%%
% FONTS
%%%%%%%%%%%%%%%%%%%%%%%%%%%%%%%%%%%%%%%%%%%%%%%%%%%%%%%%%%%%%%%%%%%%%%%%%%%%%%%%

\RequirePackage{fontspec}
\setmainfont{Carlito}


%%%%%%%%%%%%%%%%%%%%%%%%%%%%%%%%%%%%%%%%%%%%%%%%%%%%%%%%%%%%%%%%%%%%%%%%%%%%%%%%
% COLORS
%%%%%%%%%%%%%%%%%%%%%%%%%%%%%%%%%%%%%%%%%%%%%%%%%%%%%%%%%%%%%%%%%%%%%%%%%%%%%%%%

\colorlet{highlightbarcolor}{lightgray}
\colorlet{headerbarcolor}{darkgray}

\colorlet{headerfontcolor}{white}
\colorlet{accent}{awesome-red}
\colorlet{heading}{black}
\colorlet{emphasis}{black}
\colorlet{body}{black}


%%%%%%%%%%%%%%%%%%%%%%%%%%%%%%%%%%%%%%%%%%%%%%%%%%%%%%%%%%%%%%%%%%%%%%%%%%%%%%%%
% set document
%%%%%%%%%%%%%%%%%%%%%%%%%%%%%%%%%%%%%%%%%%%%%%%%%%%%%%%%%%%%%%%%%%%%%%%%%%%%%%%%


\begin{document}

\name{Sarath M}

\makeheader

\highlightbar{

    \section{Contact}
    
    \email{sarathm00@gmail.com}
    \phone{+91 9496354518}
    %\location{Some Street 1, 12345 City Name}
    \vspace{0.5em}
    %\homepage{nicokrieger.com}{https://www.nicokrieger.de}
    \github{@sarathm1}{https://github.com/sarathm1}
    %\linkedin{Nico Krieger}{https://www.linkedin.com/in/nico-krieger-6b28151b2/}
    %\orcid{0000-0003-1104-2014}{https://www.orcid.org/0000-0003-1104-2014}
    %\ads{NASA/ADS publication list}{https://ui.adsabs.harvard.edu/search/fq=\%7B!type\%3Daqp\%20v\%3D\%24fq\_database\%7D&fq\_database=database\%3A\%20astronomy&p\_=0&q=pubdate\%3A\%5B2016-01\%20TO\%209999-12\%5D\%20author\%3A(\%22Krieger\%2C\%20Nico\%22)&sort=date\%20desc\%2C\%20bibcode\%20desc}
    
    \section{Skills}
    
    \skillsection{Machine Learning Frameworks}
    \skill{Tensorflow}{4}
    \skill{Scikit-learn}{5}
    \skill{NLTK}{3}
    \vspace{0.5em}
    
    \skillsection{Visualisation \& Data Processing}
    \skill{Pandas}{5}
    \skill{Numpy}{4}
    \skill{Flask}{5}
    \skill{Bokeh}{5}
    \skill{Plotly}{4}
    \skill{Grafana}{5}
    \skill{Twitter Bootstrap}{5}
    \skill{PyQt/PySide}{5}
    \vspace{0.5em}
    
    \skillsection{Continuous Integration \& Test Automation}
    \skill{Jenkins}{4}
    \skill{Robot Framework}{5}
    \vspace{0.5em}
    
    \skillsection{Devops}
    \skill{Docker}{4}
    \skill{Ansible}{4}
    \vspace{0.5em}
    
    
    \skillsection{Other tools \& frameworks}
    \skill{Kafka}{5}
    \skill{ROS}{5}
    \skill{MQTT}{5}
    \skill{Redis}{5}
    \skill{Influxdb}{4}
    \vspace{0.5em}
    
    \skillsection{Development Tools}
    Git, SVN, Zsh, Vim, Emacs,\\VS Code, Jira
    \vspace{0.5em}
    
    %\bigskip
    
    %\section{Certificates}
    %\simpleskill{AWS certified cloud practitioner}

}
\mainbar{
    \section{Summary}
    A Machine Learning Engineer who is passionate about learning\\cutting-edge technology and solving real-world problems, currently working\\with 5 Years experience as a Specialist in Tata Elxsi
    
    \section[\faGears]{Work history}
    \job{2016 - Present}
        {Tata Elxsi, Technopark,\\Trivandrum, Kerala}
        {Specialist}
        {}
        %{Additional details for this position. Can be left empty to omit this line}
    \job{2015 - 2016}
        {Unisync Technologies, Vyttila, Kerala.}
        {Engineer}
        {}
    
    \section[\faMortarBoard]{Education}
    \job{2014}
        {Vector India Institute,\\Bangalore, Karnataka}
        {Course in Embedded Systems}
        {}
        
    \job{2010 - 2014}
        {Govt. College of Engineering\\Cherthala,\\Cochin University Of Science and Technology, Kerala}
        {B.Tech (ECE)}
        {}

    \section{Achievements, honours and awards}
    \achievement{Awarded the highest rating \textbf{"Outstanding"} in three consecutive appraisal cycles in \textit{Tata Elxsi}}
    \achievement{Final year academic project \textbf{"Hexapod"} was selected for the finals in State level
competition}
    \achievement{\textbf{Black Belt} holder in Shito-Ryu style of \textit{Karate}.}

    \section{General Skills}
    \smallskip % additional skip because tag outlines use up space
    \tag{Team Player}
    \tag{Passionate Programmer}
    \tag{Fast Learner}
    \tag{Energetic}
    \tag{Dedicated to perfection}
    
   %\medskip
   %Tags must be ordered by hand with newlines to get a nice layout, especially for long tags.
    
   \section{Wheel Chart}
   % This is taken from AltaCV
   % see https://github.com/liantze/AltaCV for details
   \wheelchart{1.5cm}{0.5cm}{% outer and inner diameter
       6/8em/accent!20/Sleep,          % comma-separated list of
       8/8em/accent!40/Daytime job,    % fraction of 24 / line length / color / label
       2/8em/accent!80/Training,          % here, the color is shades of the accent color
       3/8em/accent!60/Recovering from fighting criminals,
       5/8em/accent/Being Batman
   }
}
\makebody
\clearpage

\pagestyle{empty}
% Declare new goemetry for the second page only.
\newgeometry{margin=0.8in}

%\pagestyle{highlightmain}
%\highlightbar{}
%\mainbar{
    \section{Projects}
    \projects{2020 - 2021}{Battery Management System}{The objective of the project is to create an intelligent Battery Management System for Electric Vehicles}{Initial prototype for testing SPI communication between BMS master controller and a slave board using Raspberry Pi. Create Machine learning algorithms that predict the age and SOC of the battery cell. Creating a GUI and test-suite for the project}
  
  \projects{2018-2019}{Lidargen}{The objective of the project was to create a tool that would simulate the actual output from a LIDAR sensor. The tool takes three dimensional CAD objects as input and generates corresponding pointcloud. The User is given the freedom to place and orient the multiple meshes to create a virtual scenario}{To mathematical model the behavior of Lidar and simulate the same using Ray Casting. Designing a Qt based user interface.}
  
  \projects{2017 - 2018}{UAV Based Driver View Enhancement}{To develop a proof of concept, using a Drone to enhance the view of the Driver and cover blind spots for a larger area around the vehicle}{Designing a State Machine to control the Drone, Create a web application as a User interface using Python as a backend}
  
  \projects{2016-2017}{Pedestrian Detection \& Characterization System}{The objective of the project is to use Deep learning based intelligent algorithms for Driver Assistance and feature addition to Autonomous driving platform}%
  {%
  \begin{itemize}
  \item Development of Deep learning algorithm
 \item  Integration and testing with other modules providing additional ADAS features 
\item  Testing and deployment of the ML model in a NVidia Jetson TX1 platform
  \end{itemize}
  }
  
  \projects{2016-2017}{Vehicle Detection}{Using Deep learning algorithms to detect, classify and locate vehicles on an urban road using Camera and depth sensors}{Development of Machine Learning algorithms for object detection, Testing and validation of the machine learning models, Data Analytics to visualize the performance and debugging the application code}
  
  \projects{2014-2016}{IoT Project}{To develop a cloud based data acquisition system using Raspberry PI that communicates with Programmable Logic Controllers and sends data to Cloud}{Full responsibility of software design, implementaion and testing of the project, Deploying webserver in AWS, Communication between Raspberry Pi and PLC using Modbus Protocol}
  % \bigskip
  % This page uses the page style \texttt{highlightmain} which shows the highlight bar (gray) and the main part (white background) but omits the header. 
  % The default page style is \texttt{headerhighlightmain} with all three elements.
  % If you don't want header, nor highlight bar, use page style \texttt{\textbackslash pagestyle\{empty\}}.
  % \medskip
  % Neither main, nor highlight bar must be filled to make this template work.
  % It is possible to use a page style with the highlight bar but leave it empty by setting an empty highlightbar \texttt{\textbackslash highlightbar\{\}}.
%
  % \vspace{0.5em}
  % \subsection{Subsection 1}
  % Demonstrate subsections.
  % 
  % \subsection{Subsection 2}
  % Subsection are also bold face but a smaller font then section. They also omit the rule.
%}

%\makebody


%\clearpage
%\pagestyle{empty}
%
%\section{Publications}
%\pubforcefullwidth
%
%Demonstrate what an \texttt{\textbackslash pagestyle\{empty\}} page looks like.
%Also show off the macros for publications that uses small icons for authors, date, journal and links.
%
%Achieving a good looking spacing can be tricky. For empty pagestyles where the full width is available use \texttt{\textbackslash pubforcefullwidth} to force the publoication list to take up all the available space.
%The (relative) lengths reserved for date, journal and links can be set with the parameters \texttt{\textbackslash pubdatelength}, \texttt{\textbackslash pubjournallength} and \texttt{\textbackslash publinklength} as in \texttt{\textbackslash setlength\{\textbackslash pubdatelength\}\{0.15 \textbackslash linewidth\}}.
%\bigskip
%
%\publication
%	{The turbulent gas structure in the centers of NGC~253 and the Milky Way} % Title
%	{\textbf{N. Krieger}, A. Bolatto, E. Koch, A. Leroy, E. Rosolowsky, F. Walter, A. Wei\ss, D. Eden, R. Levy, D. Meier, E. Mills, T. Moore, J. Ott, Y. Su, S. Veilleux} % Authors
%	{2020} % Year
%	{The Astrophysical Journal Vol. 899, Issue 2, id.158} % Journal
%	{\ADS{https://ui.adsabs.harvard.edu/abs/2020ApJ...899..158K}, \arXiv{https://arxiv.org/abs/2008.02518}} % ADS & arxiv links
%
%\publication
%	{The molecular ISM in the Super Star Clusters of the starburst NGC253} % Title
%	{\textbf{N. Krieger}, A. Bolatto, A. Leroy, R. Levy, E. Mills, D. Meier, S. Veilleux, F. Walter, A. Wei\ss} % Authors
%	{2020} % Year
%	{The Astrophysical Journal Vol.897, Issue 2, id.176} % Journal
%	{\ADS{https://ui.adsabs.harvard.edu/abs/2020ApJ...897..176K}, \arXiv{https://arxiv.org/abs/2006.08262}} % ADS & arxiv links
%
%\publication
%	{The Molecular Outflow in NGC\,253 at a Resolution of Two Parsecs} % Title
%	{\textbf{N. Krieger}, A. Bolatto, F. Walter, A. Leroy, L. Zschaechner, D. Meier, J. Ott, A. Wei\ss, E. Mills, S. Veilleux, M. Gorski} % Authors
%	{2019} % Year
%	{The Astrophysical Journal Vol.881, Issue 1, article id. 43, 20 pp} % Journal
%	{\ADS{https://ui.adsabs.harvard.edu/abs/2019ApJ...881...43K}, \arXiv{https://arxiv.org/abs/1907.00731}} % ADS & arxiv links

\end{document}
